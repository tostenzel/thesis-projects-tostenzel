\section{Conclusion}
\thispagestyle{plain}  % surpress header on first page

\noindent
I analyse the uncertainty of the effect of a 500 USD subsidy on annual tuition costs for higher education on the average years of education caused by the parametric uncertainty in the model of occupational choice by \cite{Keane.1994}. The UQ has two stages. The first stage is an uncertainty analysis and the second stage is a quantitative GSA.\\

\noindent
The quantitative GSA finds that, first, the mean effect of the tuition subsidy is an increase of 1.5 years in education, and second, that the standard deviation is 0.1 years in education for the whole population. Therefore, the general parametric uncertainty is relatively small.\\

\noindent
The qualitative GSA and the conceptual analysis lead to three main results.
First, the sensitivity measures developed in \cite{ge2017extending} can not be used to rank input parameters according to their impact on the output variation because the independent EE is not unaffected by the correlations between input parameters. The thesis develops two updated measures: the correlated and uncorrelated EEs, $d_i^{c}$ and $d_i^{u}$. They are validated by the analysis of a linear test function and the economic model.

The second finding is that the sigma-normalised mean absolute correlated and uncorrelated EEs, $\mu^{*,c}_{\sigma}$ and $\mu^{*,u}_{\sigma}$, appear to be valuable screening measures. Nonetheless, I do not provide recommendations to fix specific parameters because, first, cut-off criteria for the qualitative measures, and second, the link between these measures and quantitative GSA measures remain unclear.

Third, the radial sampling scheme produces larger effects than the trajectory scheme for all input parameters. This suggests that the changes to the input parameters in both schemes are too large to capture the QoI's variation appropriately.