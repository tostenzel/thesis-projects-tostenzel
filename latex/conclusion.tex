\section{Conclusion}
\thispagestyle{plain}  % surpress header on first page

\noindent
I analyse the uncertainty in the effect of a 500 USD subsidy on annual tuition costs for higher education on the average years of education caused by the parametric uncertainty in the model of occupational choice by \cite{Keane.1994}. The UQ has two stages. The first stage is an uncertainty analysis and the second stage is a quantitative GSA.\\

\noindent
The findings of the quantitative GSA are that the mean effect is an increase of 1.5 years and that the standard deviation is 0.1 years in education for the whole population. Therefore, the general parametric uncertainty is relatively small.\\

\noindent
The qualitative GSA leads to three main results.
First, the sensitivity measures developed in \cite{ge2017extending} can not be used to rank input parameters according to their impact on the output variation because the independent EE is not neutral to correlations between input parameters.

Second, the thesis develops two update measures, the correlated and the uncorrelated EE, $d_i^{c, T}$ and $d_i^{u, T}$. The sigma-normalised mean absolute correlated and uncorrelated EEs, $\mu^{*,c}_{\sigma}$ and $\mu^{*,u}_{\sigma}$ can be valuable screening measures because they produce sensible results. Nonetheless, I do not provide recommendations to fix specific parameters because the link of these measures to quantitative GSA measures remains unclear.


Third, the radial sampling scheme produces larger effects for all input parameters than the trajectory scheme. This suggests that the input changes in both sampling schemes are too large to capture the variation in the QoI appropriately.