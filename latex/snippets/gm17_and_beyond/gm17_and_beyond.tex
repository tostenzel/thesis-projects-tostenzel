\documentclass[a4paper,12pt]{article} 

% packages and main settings
\usepackage[left=3cm, right=2cm, top=2cm, bottom=2cm]{geometry}
\usepackage[english]{babel}    
\usepackage[utf8]{inputenc}  
\usepackage[T1]{fontenc}        
\usepackage{lmodern}            
\usepackage{microtype}          
\usepackage{amsmath}
\usepackage{amsfonts, amsthm, amssymb, graphicx, booktabs}
\usepackage{bm} %bold epsilon
\usepackage{newclude}   
\usepackage{placeins}  %surpresses floating tables
\usepackage[labelfont=bf]{caption} %Figure etc steht dann in small caps 
\usepackage[labelsep=period]{caption} % dot after figure, table caption.
\usepackage[flushleft]{threeparttable} % for notes below table
\usepackage{multirow} % for table cell merge along rows
\usepackage{graphicx} % to adjust tablesize to textwidth
\usepackage{caption}  % for centered captions
\usepackage{float} % to set of autopositioning of tables
\usepackage[bottom,hang,flushmargin]{footmisc} % forces footnotes to the bottom
\usepackage{setspace}           % Fuer 1.5 fachen Zeilenabstand  
\onehalfspacing % 1.5 cm Zeilenabstand
%Bibtex
\usepackage[round,sort&compress]{natbib}

\bibliographystyle{chicago} % chicago bib style like in AER
\usepackage[hidelinks]{hyperref} % fuer links und verweise. Cleverref ist eigentlich besser. 


% Create header. The header must be surpressed for 
% every first page per section and a solution
% for the Appendix is used in the respective subfile.
\usepackage{fancyhdr}
\pagestyle{fancy}
\fancyhf{}
\chead{\nouppercase{\textit{\leftmark}}}
\cfoot{\thepage}
\renewcommand{\headrulewidth}{0pt} % no vertical line

%\usepackage{lipsum}  % check if formats work

\usepackage{afterpage} %clearpage w/o pagebreak for "header bug"

% Expectation symbol
\DeclareMathOperator*{\E}{\mathbb{E}}

% thin space, limits underneath in displays
% for strike through
\DeclareMathOperator*{\argmax}{argmax}
\newcommand*{\defeq}{\stackrel{\text{def}}{=}}
\usepackage[normalem]{ulem}
% try to use strikeout in section headers and others
\DeclareRobustCommand{\hsout}[1]{\texorpdfstring{\sout{#1}}{#1}}

% for gray table row color
\usepackage[table]{xcolor}

% decimal dot alignment in table columns
\usepackage{siunitx}

% for footnotes in table
\usepackage[flushleft]{threeparttable}

% for underbar
\newcommand{\ubar}[1]{\text{\b{$#1$}}}

\usepackage{tikz}

% Setup for urls
\usepackage{url}

\defcitealias{Respy-Stenzel.2019}{\textit{respy}}
\defcitealias{Gabler.2019}{\textit{estimagic}}
\defcitealias{Stenzel.2020}{\textit{Master's Thesis Replication Repository}}
\defcitealias{NLSY79}{NLSY79}


\usepackage{tikz}


\usepackage{enumitem}



\begin{document}

\newpage % delete after section is complete

\section{Qualitative GSA measures for functions with correlated input parameters}
\thispagestyle{plain} % suppress header on first page


\cite{Saltelli.2008}

\cite{lemaire2013structural}

\cite{Smith.2014}

\subsection{Qualitative General Sensitivity Analysis}
\noindent
Qualitative Global Sensitivity Analysis (Qualitative GSA) deals with computing measures that can rank random input parameters in terms of their importance on the function output and the variability thereof. If the measures for some input parameters are negligibly small, these parameters can be fixed so that the number of random input parameters decreases for a subsequent quantitative GSA. This pre-selection step is called Factor Fixing. The quantitative GSA then aims to determine the precise effect size of each random input parameter on the function output. The most common measures in quantitative GSA are the so-called Sobol' sensitivtiy indices. Equation 1 shows the first order index. It is the share of the variance in the function output induced by exclusively one single input parameter $X_i$ of the variance induced by all random input parameters $X_1, X_2, ..., X_k$.


\begin{align}
S_i = \frac{\text{Var}_i[Y|X_i ]}{\text{Var}[Y]}
\end{align}

\noindent
Equation 2 shows the total order index. This measure is equal to the first order index except of that its numerator includes the variance in the function output that is induced by changes in the other input parameters $X_{\sim i}$, caused by interactions with the variation in $X_i$.

\begin{align}
S_{i}^T = \frac{\E_{\sim i}[\text{Var}_{i}[Y|\bold{X_{\sim i}]]}}{\text{Var}[Y]}
\end{align}

\noindent
Computing these measures requires many function evaluations, even if an estimator is used as a shortcut. The more time-intense one function evaluation is, the more utility provides the aforementioned Factor Fixing based on qualitative measures. The most commonly used measures in qualitative GSA is the Elementary Effect (EE), $\mu$, the absolute Elementary Effects, $\mu^*$, and the standard deviation of the Elementary Effect, $\sigma$. The Elementary Effect is given by the mean of a number of function derivatives with respect to one input parameter. The "change in", or the "step of" the input parameter, denoted by $\Delta$, has not to be infinitesimally small. The only restriction is that $X_i + \Delta$ is in the domain of $X_i$. The derivation, also called the individual Elementary Effect, is denoted by
\begin{align}
d^{(j)} =  \frac{Y(\bold{X_{\sim i}^{(j)}}, X_i^{(j)} + \Delta^{(i,j)})}{\Delta^{(i,j)}},
\end{align}
where $j$ is an index for the number of observations of $X_i$.
Then, the Elementary Effect is given by

\begin{align}
\mu = \frac{1}{r} \sum_{j=1}^{r} d^{(j)}.
\end{align}
\noindent
The absolute Elementary Effects, $\mu^*$ is used to prevent observations of opposite sign to cancel each other out.

\begin{align}
\mu^* = \frac{1}{r} \sum_{j=1}^{r} \big| d^{(j)} \big|.
\end{align}
\noindent
In Equation 4 and 5, $r$ is the number of parameter draws with index $j$. Step $\Delta^{(j)}$ may or may not vary depending on the sample design that is used to draw the input parameters. These measures (together) are used to proxy the total Sobol' indices that contains the parameter-specific interactions with all other parameter in Equation(2). The total Sobol' index is the relevant one because it also contains interactions that may be important. If the qualitative measures are close to 0 for one particular parameter, its variation can be rendered as irrelevant for the variation in the function output (given there are parameters with measures substantially different from 0).

\subsection{Sampling Schemes}

According to several experiments by \cite{campolongo2011screening} using common test functions, the best design is the radial design (\cite{saltelli2002making}) and the most commonly used is the trajectory design (\cite{Morris.1991}).
Both designs are comprised by a $(k + 1) \times k$-dimensional matrix. The elements are generated in $[0,1]$. Afterwards, they can be transformed to the distributions of choice. The columns represent the input parameters and each row is a complete input parameter vector. To compute the qualitative measures, a set of multiple matrices, or subsets, of input parameters has to be generated.\\

\noindent
A matrix in radial design is generated in the following way: Draw a vector of length $2k$ from a quasi-random sequence. The first row, or parameter vector, is the first half of the sequence. Then, copy the first row to the remaining $k$ rows. For each row $k'$ of the remaining 2, ..., $k+1$ rows, replace the $k'$-th element by the $k'$-th element of the second half of the vector. This generates a matrix of the following form:
\begin{align}
\underset{(k+1)\times k}{\bold{R}} =
\begin{pmatrix}
a_1 & a_2 & ... & a_k \\
\bold{b_1} & a_2 & ... & a_k \\
a_1 & \bold{b_2} & ... & a_k \\
\vdots & \vdots & \vdots & \vdots\\
a_1 & a_2 & ... & \bold{b_k}
\end{pmatrix}
\end{align}
\noindent
Note here, that each column consists only of the first row element, except of for one row.
From this matrix, one individual EE can be obtained for each parameter $X_i$. This is achieved by using the $(i+1)$-th row as function argument for the minuend and the first row as subtrahend in the formula for the individual EE. Then, $\Delta^{(i,j)} = b_i^{(j)} - a_i^{(j)}$. Tailored to the radial scheme, the derivation can be re-formulated as in Equation (3), where $j$ is the number of radial subsamples and $i$ represents the input parameter $X_i$.
\begin{align}
d^{(j)}_i =  \frac{Y(\bold{a_{\sim i}^{(j)}}, b_i^{(j)}) - Y(\bold{a})}{b_i^{(j)} - a_i^{(j)}} = \frac{Y(\bold{R_{i+1,*}}) -  Y(\bold{R_{1,*}})}{b_i^{(j)} - a_i^{(j)}}.
\end{align}
If the number of radial subsamples is high, the quasi-random sequence lead to a good and fast coverage of the input space (compared to a random sequence). The quasi-random sequence considered here is the Sobol' sequence. This sequence is comparably succesful in covering the unit hypercube, but also conceptually more involved. Therefore, its presentation is beyond the scope of this work. Since this sequence is quasi-random, the sequence has to be drawn at once for all sets of radial matrices.\\

\noindent
Next, I present the trajectory design. As we will see, it leads to a relatively representative coverage for a very small number of subsamples but also to frequent repetitions of similar draws.
In this outline, I skip the equations that generate a trajectory and present the method verbally.
There are different forms of trajectories. Here, I focus on the version presented in \cite{Morris.1991} that yields to equiprobable elements. The first step is to decide the number $p$ of grid points in interval $[0,1]$. Then, the first row of the trajectory is composed of the lower half value of these grid points. Now, fix $\Delta = p/[2(p-1)]$. This function implies, that adding $\Delta$ to the lowest point in the lowest half results in the lowest point of the upper half of the grid points, and so on. The rest of the rows is constructed by, first, copying the row one above and, second, by adding $\Delta$ to the $k$-th element of the $k+1$-th row. The implied matrix scheme is depicted below.
\begin{align}
\underset{(k+1)\times k}{\bold{T}} =
\begin{pmatrix}
a_1 & a_2 & ... & a_k \\
\bold{b_1} & a_2 & ... & a_k \\
\bold{b_1} & \bold{b_2} & ... & a_k \\
\vdots & \vdots & \vdots & \vdots\\
\bold{b_1} & \bold{b_2} & ... & \bold{b_k}
\end{pmatrix}
\end{align}
\\

\noindent
In contrary to the radial scheme, each $b_i$ is copied to the subsequent row. Therefore, the EEs have to be determined by comparing each row with the row above instead of with the first row.
Importantly, two random transformations are common. These are randomly switching rows and randomly interchanging the $i$-th column with the $(k-i)$-th column and then reversing the column. The first transformation is skipped as it does not add additional coverage and because we need the stairs-shaped design to facilitate later transformations which account for correlation between input parameters. The second transformation is adapted because it is important to also have negative steps and because it does sustain the stairs shape. Yet, this implies that $\Delta^{(i,j)}$ is also parameter- and trajectory-specific. Let $f$ and $h$ be additional indices representing the input parameters. The derivation formula is adapted to the trajectory design as follows:\footnote{In contrary to most authors, I also denote the step as a subtraction instead of $\Delta$ when referring to the trajectory design. This provides additional clarity.}

\begin{align}
d^{(j)}_i =  \frac{Y(\bold{b_{f \leq i}^{(j)}}, \bold{a_{h>i}^{(j)}}) - Y(\bold{b_{f<i}^{(j)}}, \bold{a_{h \geq i}^{(j)}})}{b_i^{(j)} - a_i^{(j)}} = \frac{Y(\bold{T_{i+1,*})} -  Y(\bold{T_{i,*}})}{b_i^{(j)} - a_i^{(j)}}.
\end{align}
The trajectory design involves first, a fixed grid, and second and more importantly, a fixed step $\Delta$. s.t. $\{\Delta\} = \{\pm \Delta\}$. This implies less step variety and less space coverage vis-á-vis the radial design for more than a small number of draws.\\

\noindent
So far, we have only considered draws in [0,1]. For uncorrelated input parameters from arbitrary distributions with well-defined cumulative distribution function, $\Phi$, one would simply evaluate each element (of course, potentially including the addition of the step) by the inverse cumulative distribution function, $\Phi^{-1}$, of the respective parameter. A bit of intuition is, that $\Phi$ maps the sample space to [0,1]. Hence $\Phi^{-1}$ can be used to transform random draws in [0,1] to the sample space of the arbitrary distribution. This is a basic example of so-called inverse transform sampling which we will recall in the next section.



\subsection{The approach for correlated input parameters in \cite{ge2017extending}}

This section describes the incomplete approach by \cite{ge2017extending} to extend the EE-based measures to input parameters that are correlated. Their main achievement is to outline a transformation of samples in radial and trajectory design that incorporates the correlation between the input parameters. This implies, that the trajectory and radial samples cannot be written as in Equation (6) and Equation (8). The reason is that the correlations of parameter $X_i$, to which step $\Delta^i$ is added, implies that all other parameters $\bold{X_{\sim i}}$ in the same row with non-zero correlation are changed as well. Therefore, the rows cannot be denoted and compared as easily by $a$'s and $b$'s as in Equation (6) and (8). Transforming these matrices allows to re-define the EE-based measures accordingly, such that they sustain the main properties of the ordinary measures for uncorrelated parameters. The property is to be a function of the mean derivative. Yet, \cite{ge2017extending} fail to fully develop these measures. I will explain how their measures lead to arbitrary results for correlated input parameters. This section covers their approach in a simplified form, focussing on normally distributed input parameters, and presents their measures.\\

\noindent
The next paragraph deals with developing a recipe for transforming draws $\bold{u} = \{u_1, u_2, ..., u_k\}$ from $[0,1]$ for an input parameter vector to draws $\bold{x} = \{x_1, x_2, ..., x_k\}$ from an arbitrary joint normal distribution. We will do this in three steps. 

For this purpose, let $\bold{\Sigma}$ be a non-singular variance-covariance matrix and let $\pmb{\mu}$ be the mean vector. The $k$-variate normal distribution is denoted by $\mathcal{N}_k(\pmb{\mu}, \bold{\Sigma})$. \\

\noindent
Creating potentially correlated draws $\bold{x}$ from $\mathcal{N}_k(\pmb{\mu}, \bold{\Sigma})$ is simple. Following \cite{gentle2006random}, page 197, this can be done the following way: Draw a $k$-dimensional row vector of i.i.d standard normal deviates from the \textit{univariate} $N_1(0,1)$ distribution, such that  $\bold{z} = \{z_1, z_2, ..., z_k\}$, and compute the Cholesky decomposition of $\Sigma$, such that $\bold{\Sigma} = \bold{T^T T}$. The lower triangular matrix is denoted by $\bold{T'}$. Then apply the operation in Equation (10) to obtain the correlated deviates from $\mathcal{N}_k(\pmb{\mu}, \bold{\Sigma})$.
\begin{align}
\bold{x} = \pmb{\mu} + \bold{T^T z^T} 
\end{align}
Intuition for the mechanics  that underlie is provided in Appendix A (not contained this draft). \\

\noindent
The next step is to understand that we can split the operation in Equation (10) into two subsequent operations. For this, let $\pmb{\sigma}$ be the vector of standard deviations and let $\bold{R}$ be the correlation matrix of $\bold{x}$.

The first operation is to transform the standard deviates $\bold{z}$ to correlated standard deviates $\bold{z_c}$ by using $\bold{z_c}=\bold{Q^T z^T}$. In this equation, $\bold{Q^t}$ is the lower matrix from the Cholesky decomposition $\bold{R_k}=\bold{Q^T Q}$. This is equivalent to the above approach in \cite{gentle2006random} for the specific case of the multivariate standard normal distribution $\mathcal{N}_k(0, R_k)$. This is true because for multivariate standard normal deviates, the correlation matrix is equal to the covariance matrix.

The second operation is to scale the correlated standard normal deviates: $\bold{z}=\bold{z_c}\pmb{\sigma} + \pmb{\mu}$.\\

\noindent
The last step is to recall the inverse transform sampling method. Therewith we can transform the input parameter draws $\bold{u}$ to uncorrelated standard normal draws $\bold{z}$. Then we will continue with the two operation in the above paragraph. The transformation from $\bold{u}$ to $\bold{z}$ is denoted by $ F^{-1}(\Phi)$ and summarized by the following three points:


\[
\left.\parbox{0.5\textwidth}{%
\begin{enumerate}[label=\bfseries Step \arabic*:,leftmargin=*,labelindent=5em]
	\item $\bold{z} = \pmb{\Phi^{-1}({u})}$
    \item $\bold{z_c} = \bold{z Q^T}$
    \item $\bold{x} = \pmb{\mu} + \bold{z_c(i)}\pmb{\sigma(i)}$
\end{enumerate}
}\right\}F^{-1}(\Phi)
\]


\noindent
The above procedure is equivalent to an inverse Rosenblatt transformation, a linear inverse Nataf transformation\footnote{For the first two transformations, see \cite{lemaire2013structural}, page 78 - 113} and a linear Guassian copula. These concepts can be used to transform deviates in [0,1] to the sample space of arbitrary distribution by using the properties sketched above under varying assumptions. Notes on this relation are provided in Appendix B (not contained in this draft).\\

\indent
The one most important thing to understand is that the transformation comprised by the three steps listed above is not unique for correlated input parameters. Rather, the transformation changes with the order of parameters in vector $\bold{u}$ (see formal definition of the Rosenblatt transformation). This can seen from the lower triangular matrix $\bold{Q^T}$. For the next equation, let $R_k = (\rho_{ij})_{ij=1}^k$ and sub-matrix $R_h = (\rho_{ij})_{ij=1}^h$. Also let $\rho_i^{*,j} = (\rho_{1,j}, \rho_{1,j}, ..., \rho_{i-1,j})$ for $j \geq i$, such that $\rho_i=\rho_i^{*,j}$. Then, following \cite{madar2015direct}, 

\begin{align}
\bold{Q^T} =
\begin{pmatrix}
\\ 1 & 0 & 0 & ... & 0
\\\rho_{1,2} & \sqrt{1-\rho_{1,2}^2} & 0 & ... & 0
\\ \rho_{1,3} & \frac{\rho_{2,3}-\rho_{1,2}\rho_{1,3}}{\sqrt{1-\rho_{1,2}^2}} & \sqrt{1-\rho_{3}R^{-1}_2\rho_{3}^T} & ... & 0
\\\vdots & \vdots & \vdots & \vdots & \vdots
\\ \rho_{1,k} & \frac{\rho_{3,k}-\rho_{3}^{*,k}R^{-1}_2\rho_{3}^T}{\sqrt{1-\rho_{3}R^{-1}_2\rho_{3}^T}} & ... &
... & \sqrt{1-\rho_{k}R^{-1}_2\rho_{k}^T}
\end{pmatrix}.
\end{align}
This implies, looking at Step 2, that the order of the uncorrelated standard normal deviates $\pmb{z}$ implies a hierarchy amongst the correlated deviates $\pmb{z_c}$, such that: The first parameter is not subject to the correlation, the second parameter is subject to the correlation with the first parameter, the third parameter is subject to the parameters before, etc. Therefore, typically $F^{-1}(\Phi)(\bold{u}) \neq F^{-1}(\Phi)(\bold{u'})$, where $\bold{u'}$ denotes $\bold{u}$ in a different order.

Coming back to the EE-based measures for correlated inputs, \cite{ge2017extending} construct two measures for each measure for uncorrelated inputs. One half of these measures is based on the so-called independent Elementary Effects and the other half on the full Elementary Effects. The difference between these measures is that the first excludes and the second includes correlation. It is important to have the independent Elementary Effects, because it is possible that the correlation decreases the measure of one input parameter (close to 0). However, if the independent-EE-based measures are not close to zero, $X_i$ is still important. Additionally, fixing this parameter might possibly change the full-EEs of $X_{\sim i}$. Another notable is that the full Elementary Effect is design to include the additional effect of the step in the two sampling schemes on the other parameters

Considering the impact of the parameter order, \cite{ge2017extending} apply the following reordering transformation $\mathcal{T}$ right before $F^{-1}(\Phi)$ to each trajectory to obtain the minuend of the full EE:

\begin{align}
\mathcal{T}({\bold{R}}) =
\mathcal{T}
\begin{pmatrix}
a_1 & a_2 & ... & a_k \\
\bold{b_1} & a_2 & ... & a_k \\
\bold{b_1} & \bold{b_2} & ... & a_k \\
\vdots & \vdots & \vdots & \vdots\\
\bold{b_1} & \bold{b_2} & ... & \bold{b_k}
\end{pmatrix}
=
\begin{pmatrix}
a_1 & a_2 & ... & ... &  a_k \\
\bold{b_1} & a_2 & ... & ... &  a_k \\
\bold{b_2} & a_3 & ... & ... &  \bold{b_1} \\
\vdots & \vdots & \vdots & ... &  \vdots\\
\bold{b_k} & \bold{b_{1}} & ... & ... &  \bold{b_{k-1}}
\end{pmatrix}
\end{align}


The reordering transformation for the minuend of the independet EE is.



\begin{align}
\mathcal{T}({\bold{R}}) =
\mathcal{T}
\begin{pmatrix}
a_1 & a_2 & ... & a_k \\
\bold{b_1} & a_2 & ... & a_k \\
\bold{b_1} & \bold{b_2} & ... & a_k \\
\vdots & \vdots & \vdots & \vdots\\
\bold{b_1} & \bold{b_2} & ... & \bold{b_k}
\end{pmatrix}
=
\begin{pmatrix}
a_1 & a_2 & ... &  ... & a_k \\
a_2 & ... & ... &    a_k & \bold{b_1} \\
a_3 & ... &  a_k & \bold{b_1}  & \bold{b_2}\\
\vdots & \vdots & \vdots & \vdots & \vdots\\
\bold{b_1} & \bold{b_2} & ... &   ... & \bold{b_k}
\end{pmatrix}
\end{align}


- Here I run short of time and will cut it short to get to my own measures and some comments on the implementation.\\

- The main idea of the reordering is for the minuend in $EE_full$ to move the newly altered $a+\Delta$ to the FRONT and the old $a+\Delta$ to the back. Therewith, only the $\Delta$ in the $a+\Delta$ is included in the on-sided correlation of all other parameters thereafter. The subtrahent of each row is the very same row except of that $\Delta$ is not added to the first element. \\

- The main idea of the reordering is for the minuend in $EE_ind$ to move the newly altered $a+\Delta$ to the BACK and the old $a+\Delta$ in front of the BACK. Therewith, only no $\Delta$ included in the on-sided correlation of all other parameters after the front. The subtrahent of each row is the very same row except of that $\Delta$ is not added to the first element. \\

- The only difference from the radial design to the trajectory design is, that the matrix for the subtrahend is only comprised of the first rows. Yet, the order of the first rows is also changed in the same way as for the trajectory design! \\

These four transformations are applied to each trajectory or radial sample. Then, $F^{-1}(\Phi)(\bold{u'})$ is applied to each row - using the covariance matrix and mean vector according to the new order of the parameter vector for each row. \\

Afterwards, the reorder-transformation is reversed.



Then, using the respective rows \cite{ge2017extending} apply the formula for $d^{(j)}$ to obtain the individual Elementary Effects. \\

{\color{red}
The mistake is, that they transform $\Delta$ and $(b_i - a_i)$ multiple times. $F^{-1}(\Phi)$ for normally distributed parameters involves the non-linear transformation by $\Phi$ and the scaling by $\sigma$. This transformation must also be executed for the step in the denominator for both Effects. Additionally, the step in the denominator of the independent effect has to be multiplied by the last element of the lower triangular matrix from the Cholesky decomposition of the correlation matrix, i.e. ${Q^T}_{k,k}$. If this is not done, $EE_full$ is non-linearily distorted and $EE_ind$ is non-linearily distorted and distorted such that the effects for parameters at later positions in the parameter vector are decreased.
The correct denominator for both $EE_{full}$ is $F^{-1}(\Phi(b)) - F^{-1}(\Phi(a))$. The correct denominator for both $EE_{ind}$ is ${Q^T}_{k,k}(F^{-1}(\Phi(b)) - F^{-1}(\Phi(a)))$

Remarks on the implementation. I can implement all results by GM, see the notebooks in the README at branch $`replication_gm17`$: $https://github.com/HumanCapitalAnalysis/thesis-projects-tostenzel/tree/replication_gm17$

The authors measures do not work for uncorrelated parameters because they are distored.
My measures work for these cases. I solved one test function from \cite{Smith.2014}, page 335 which is very similar to Test Case 1 and 2 in \cite{ge2017extending} (correlated normal). I also solved the large test function from \cite{Saltelli.2008}, page 123 - 239 (uncorrelated uniform). These can be found in $test_screening_measures.py$ on the master branch.

My results for Test Case 1 and 2 are correct: The EE-full is the sum of coefficients times the correlation of one parameter with all other parameters including itself. The EE-ind is only its coefficient. The SD is always zero as the effect of a linear function is always the same of its domain, see \cite{Smith.2014}, page 335.


}







\newpage
\bibliography{../../bibliography/literature}

\end{document}