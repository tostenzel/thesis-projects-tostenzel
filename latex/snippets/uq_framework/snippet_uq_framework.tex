\documentclass[a4paper,12pt]{article} 

% packages and main settings
\usepackage[left=3cm, right=2cm, top=2cm, bottom=2cm]{geometry}
\usepackage[english]{babel}    
\usepackage[utf8]{inputenc}  
\usepackage[T1]{fontenc}        
\usepackage{lmodern}            
\usepackage{microtype}          
\usepackage{amsmath}
\usepackage{amsfonts, amsthm, amssymb, graphicx, booktabs}
\usepackage{bm} %bold epsilon
\usepackage{newclude}   
\usepackage{placeins}  %surpresses floating tables
\usepackage[labelfont=bf]{caption} %Figure etc steht dann in small caps 
\usepackage[labelsep=period]{caption} % dot after figure, table caption.
\usepackage[flushleft]{threeparttable} % for notes below table
\usepackage{multirow} % for table cell merge along rows
\usepackage{graphicx} % to adjust tablesize to textwidth
\usepackage{caption}  % for centered captions
\usepackage{float} % to set of autopositioning of tables
\usepackage[bottom,hang,flushmargin]{footmisc} % forces footnotes to the bottom
\usepackage{setspace}           % Fuer 1.5 fachen Zeilenabstand  
\onehalfspacing % 1.5 cm Zeilenabstand
%Bibtex
\usepackage[round,sort&compress]{natbib}

\bibliographystyle{chicago} % chicago bib style like in AER
\usepackage[hidelinks]{hyperref} % fuer links und verweise. Cleverref ist eigentlich besser. 


% Create header. The header must be surpressed for 
% every first page per section and a solution
% for the Appendix is used in the respective subfile.
\usepackage{fancyhdr}
\pagestyle{fancy}
\fancyhf{}
\chead{\nouppercase{\textit{\leftmark}}}
\cfoot{\thepage}
\renewcommand{\headrulewidth}{0pt} % no vertical line

%\usepackage{lipsum}  % check if formats work

\usepackage{afterpage} %clearpage w/o pagebreak for "header bug"

% Expectation symbol
\DeclareMathOperator*{\E}{\mathbb{E}}

% thin space, limits underneath in displays
% for strike through
\DeclareMathOperator*{\argmax}{argmax}
\newcommand*{\defeq}{\stackrel{\text{def}}{=}}
\usepackage[normalem]{ulem}
% try to use strikeout in section headers and others
\DeclareRobustCommand{\hsout}[1]{\texorpdfstring{\sout{#1}}{#1}}

% for gray table row color
\usepackage[table]{xcolor}

% decimal dot alignment in table columns
\usepackage{siunitx}

% for footnotes in table
\usepackage[flushleft]{threeparttable}

% for underbar
\newcommand{\ubar}[1]{\text{\b{$#1$}}}

\usepackage{tikz}

% Setup for urls
\usepackage{url}

\defcitealias{Respy-Stenzel.2019}{\textit{respy}}
\defcitealias{Gabler.2019}{\textit{estimagic}}
\defcitealias{Stenzel.2020}{\textit{Master's Thesis Replication Repository}}
\defcitealias{NLSY79}{NLSY79}


\usepackage{tikz}


\begin{document}

\newpage % delete after section is complete

\section{Uncertainty Quantification Framework}
\thispagestyle{plain} % surpress header on first page

This section consists of three parts. The first part gives an overview of UQ and introduces the basic notation. The second part describes the subdiscipline sensitivity analysis. After general remarks, this part, in turn, is divided in quantitative and qualitative GSA. I explain the most common measures for both levels of GSA and how they relate. The third part concludes with remarks on the role of correlated input parameters.

\subsection{Overview of Uncertainty Quantification}
Model-based forecasting includes two main steps (\cite{Smith.2014})\footnote{See page ix.}: The first step is the calibration. In this step, the input parameters of the model are estimated. The second step is the prediction. The prediction contains the model evaluation at the estimated parameters to make statements about the future. These statements are made in a probabilistic way. Thereby, the uncertainty of these statements is emphasised.\footnote{The general procedure of model-based forecasting can also include other steps. However, steps like model validation and model verification can also be viewed as belonging to the analysis of the so-called model uncertainty. The concept of model uncertainty is briefly explained in the next paragraph.}\\
\newline
There are four sources of uncertainty in modern forecasting that are based on complex computational models (\cite{Smith.2014})\footnote{See page 4-7.}. The first source, the model uncertainty, is the uncertainty about whether the mathematical model represents the reality appropriately.\footnote{However, apparently there are not many powerful instruments to evaluate and improve the model uncertainty except comparing statements derived from the model to the data and then improving it where appropriate.} The second source, the input uncertainty, is the uncertainty about the size of the input parameters of the model. The third one, the numerical uncertainty, comes from potential errors and uncertainties introduced by the translation of a mathematical to a computational model. The last source of uncertainty, the measurement uncertainty, is the accuracy of the experimental data that is used to approximate and calibrate the model.

The thesis deals with the second source of uncertainty, the input uncertainty. In my view, this is the source for which UQ offers the most and also the strongest instruments. This results from the fact that the estimation step produces standard errors as basic measures for the variation or uncertainty in the input parameter estimates. These can then be used to compute a variety of measures for the impact of the input uncertainty on the variation in the model output.\\
\newline
The following explains the basic notation. It is essential to define the quantity that one wants to predict with a model. This quantity is called the quantity of interest (henceforth QoI) and is denoted by $Y$. For instance, the QoI in the thesis is the impact of a 500 USD tuition subsidy for higher education on average schooling years. The uncertain model parameters $X_1, X_2, ..., X_k$ are denoted by vector $\bold{X}$. The function that computes QoI $Y$ by evaluating a  model and, if necessary, post-processing the model output is denoted by $f(X_1, X_2, ..., X_k)$. Thus,
\begin{align}
Y = f(\bold{X}).
\end{align}
Large-scale UQ applications draw from various fields such as probability, statistics, analysis, and numeric. These disciplines are used in a combined effort for parameter estimation, surrogate model construction, parameter selection, uncertainty analysis, LSA, and GSA, amongst others. Drawing mainly from \cite{Smith.2014}\footnote{See page 8-10.}, I briefly sketch the first four components. The last two components, local and especially global sensitivity analysis, are discussed more extensively thereafter.

Parameter estimation covers the calibration step. There is a large number of estimation techniques for various types of models. The thesis uses a maximum likelihood approach, as detailed in the model section and in Appendix C.

If the run time of a model is too long to compute the desired UQ measures, surrogate models are constructed to substitute the original model $f$ (\cite{mcbride2019overview}). These surrogate models are functions of the model input parameters which are faster to evaluate. The functions are also called interpolants because they are computed from a random sample of input vectors, drawn from the input distribution and evaluated by the model. Typically, a surrogate model is computed by minimizing a distance measure between a predetermined type of function and the model evaluations at the sample points. Therefore, the surrogate model interpolates this sample. Some specifications, like orthogonal polynomials, have properties which can simplify the computation of UQ measures tremendously (\cite{xiu2010numerical}).

Another way to reduce the computation time, not directly of the model but of UQ measures, is to reduce the number of uncertain input parameters as part of a parameter selection. The decision which parameters to fix is made based on sensitivity measures. This is called screening or factor fixing (\cite{Saltelli.2008})\footnote{See page 33-34.}. This point will be taken up again after this overview.

Uncertainty analysis is the core of the prediction step. It comprises two steps. The first step is the  construction of the QoI's probability distribution by propagating the input uncertainty through the model. For instance, this can be achieved by evaluating a sample of random input parameters (as also required for the construction of a surrogate model). The second step is the computation of descriptive statistics like the probabilities for a set of specific events in the QoI range using this distribution. Both steps are conceptually simple. The construction of the probability distribution is also important for designing subsequent steps like a sensitivity analysis. For example, if the distribution is unimodal and symmetric, then variance-based UQ measures are meaningful. If the distribution has a less tractable, for instance a bimodal shape, then density-based measures are better suited (\cite{plischke2013global}).


\subsection{Sensitivity Analysis}


This section draws largely from \cite{Saltelli.2004} and \cite{Saltelli.2008}.
They define sensitivity analysis as "the study of how uncertainty in the output of a model (numerical or otherwise) can be apportioned to different sources of uncertainty in the model input" \citeauthor{Saltelli.2004} (2004, page 42). This apportioning implies a ranking of input parameters in terms of their importance for the model ouput. \citeauthor{Saltelli.2004} (2004, page 52) define the most important parameter as "the one that [if fixed to its true, albeit unknown value]
would lead to the greatest reduction in the variance of the output Y. Therefore, a factor is not important if it influences the output Y directly but rather its variance.\\

\noindent
Sensitivity analysis includes different objectives. These have to be determined at first because the choice of methods depends on these objectives. Typically, the main and final goal is factor prioritisation. This is the aforementioned ranking of input parameters in terms of their importance. This ranking can be used to concentrate resources on the data acquisition and estimation of a subset of parameters. The methods that meet the demands of factor prioritisation best are called quantitative. These typically require the highest computational effort.

There are multiple other objectives. The one additional objective featured in this thesis is screening/factor fixing. It is basically the same as factor prioritisation except that it only aims to identify the input parameters that can be fixed at a given value without significantly reducing the output variance. Therefore, it focuses on the lowest parameters in the potential importance ranking. The reason why one would purse factor fixing instead of factor prioritisation is computational costs. As factor fixing generates less information than factor prioritisation, less powerful methods can be applied. These methods require less computational resources and are called qualitative. Factor fixing can be used to prepare factor prioritisation for models that are more costly to evaluate. In this sense it serves the same purpose as surrogate modelling.\\

\noindent
Another important distinction is local versus global sensitivity analysis. It essentially refers to the applied methods. In fact, the definition by \cite{Saltelli.2004} is already tailored to a global sensitivity analysis. In contrast to the given definition, "until quite recently, sensitivity analysis was [...] defined as a local measure of the effect of a given input on a given output" \citeauthor{Saltelli.2004} (2004, page 42). This old definition differs from the definition used here in two aspects. First, it emphasises the level of output rather than its variance. Second, it describes the measure as a local one. The drawback of this approach becomes clear by considering an example of a local sensitivity measure. This measure is the so-called system derivate $D_i = \frac{\partial Y}{\partial X_i}$ (\cite{rabitz1989systems}). The derivative is typically computed at the mean, $\overline{X_i}$, of the estimate for $X_i$. $D_i$ is a so-called one-at-a-time measure because it changes only one factor. It has the following four drawbacks: First, it does not account for interaction between multiple input parameters because it is one-at-a-time. Second, If the model derivation is not analytical, the choice of the (marginal) change in $X_i$ is arbitrary. Third, the local derivative at $\overline{X_i}$ is only representative for the whole sample space of a random input if the model is linear in $X_i$. Fourth, the measure does not relate to the output variance $Var(Y)$. For these reasons, the fields, its definition and its methods have evolved beyond the notion of local sensitivity analysis. Yet, until recently, the main part of applications in different fields, such as physics (\cite{Saltelli.2004})\footnote{See page 42.} and economics (\cite{Harenberg.2019}), still use local measures.


\subsubsection{Quantitative Global Sensitivity Analysis}

The quantitative GSA aims to determine the precise effect size of each random input parameter and its variation on the output variation. The most common measures in quantitative GSA are the Sobol' sensitivity indices. Equation (\ref{eq:gen_sobol}) gives the general expression for the first order index. Let $\text{Var}_{X_i} (Y|X_i)$ denote the variance of the model output $Y$ conditional on input parameter $X_i$. Then,

\begin{align} \label{eq:gen_sobol}
S_i = \frac{\text{Var}_{X_i}(Y|X_i)}{\text{Var}(Y)}.
\end{align}

\noindent
The equation becomes clearer with the following equivalent expression in Equation (\ref{eq:spec_sobol}).
For this purpose, let $\sim i$ denote the set of indices except $i$. The expectation of $Y$ for one specific value of $X_i$ equals the average of the model evaluations from a sample, $\pmb{\chi_{\sim i}}$,  of $\bold{X_{\sim i}}$ and a given value
$X_i = x_i^*$. Then, we use $\E[f(X_i = x_i^*,\pmb{\chi_{\sim i}} )] \defeq \E_{\bold{X_{\sim i}}} [Y|X_i ]$ to write the first-order Sobol' index as the variance of $\E_{\bold{X_{\sim i}}} [Y|X_i ]$ over all $x_i^*$s as


\begin{align} \label{eq:spec_sobol}
S_i = \frac{\text{Var}_{X_i}\big( \E_{\bold{X_{\sim i}}} [Y|X_i ]\big)}{\text{Var}(Y)}.
\end{align}


\noindent
The first-order index does not include the additional variance in $Y$ that may occur from the interaction of $\bold{X_{\sim i}}$ with $X_i$. This additional variance is included in the total-order Sobol' index given by Equation (\ref{eq:tot_sobol}). It is the same expression as in Equation (\ref{eq:spec_sobol}) except that the positions for $X_i$ and $\bold{X_{\sim i}}$ are interchanged. Conditioning on $\bold{X_{\sim i}}$ accounts for the inclusion of the interaction effects of $X_i$.


\begin{align} \label{eq:tot_sobol}
S_{i}^T = \frac{\text{Var}_{\bold{X_{\sim i}}}\big( \E_{\sim i}[Y|\bold{X_{\sim i}]} \big)}{\text{Var}(Y)}
\end{align}

\noindent
Computing these measures requires many function evaluations, even if an estimator is used as a shortcut (\cite{Saltelli.2004})\footnote{See page 124 -149.}. The more time-intense one function evaluation is, the more utility provides the aforementioned factor fixing based on qualitative measures. 


\subsubsection{Qualitative Global Sensitivity Analysis}


Qualitative GSA deals with the computation of measures that can rank random input parameters in terms of their impact on the function output and the variability thereof. This is done to a degree of accuracy that allows to distinct between influential and non-influential parameters. If the measures for some input parameters are negligibly small, these parameters can be fixed so that the number of random input parameters decreases for a subsequent quantitative GSA. This section explains the qualitative measures and the trade-off between computational costs and accuracy. \\

\noindent
The most commonly used measures in qualitative GSA is the mean EE, $\mu$, the mean absolute EEs, $\mu^*$, and the standard deviation of the EEs, $\sigma$. The EE of $X_i$ is given by one individual function derivative with respect to $X_i$. The "change in", or the "step of" the input parameter, denoted by $\Delta$. The only restriction is that $X_i + \Delta$ is in the sample space of $X_i$. The Elementary Effect, or derivative, is denoted by
\begin{align}
d_i^{(j)} =  \frac{Y(\bold{X_{\sim i}^{(j)}}, X_i^{(j)} + \Delta^{(i,j)})}{\Delta^{(i,j)}},
\end{align}
where $j$ is an index for the number of $r$ observations of $X_i$.
Note, that the EE, $d_i^{(j)}$ is equal to the aforementioned local measure, the system derivate $S_i = \frac{\partial Y}{\partial X_i}$, except that the value $\Delta$ has not to be infinitesimally small. To offset the third drawback of $d_i$ or $S_i$, that base vector $X_i$ does not represent the whole input space, one computes the mean EE, $\mu_i$, based on a random sample of $X_i$ from the input space. The second drawback, that interaction effects are disregarded, is also offset because elements $X_{\sim i}$ are also resampled for each new $X_i$. This measure is given by

\begin{align}
\mu_i = \frac{1}{r} \sum_{j=1}^{r} d_i^{(j)}.
\end{align}
\noindent
Thus, $\mu_i$ is the global version of $d_i^{(j)}$. Then, the standard deviation of the EEs writes $\sigma_i = \frac{1}{r} \sum_{j=1}^{r} (d_i^{(j)} - \mu_i)$. The mean absolute EE, $\mu_i^*$ is used to prevent observations of opposite sign to cancel each other out:

\begin{align}
\mu_i^* = \frac{1}{r} \sum_{j=1}^{r} \big| d_i^{(j)} \big|.
\end{align}
\noindent
Step $\Delta^{(i,j)}$ may or may not vary depending on the sample design that is used to draw the input parameters.\\


\noindent
One last improvement is provided in \cite{Smith.2014}\footnote{See page 332.}. That is, the scaling of $\mu_{i}^*$ by $\frac{\sigma_{X_i}}{\sigma_Y}$. This measure is called the sigma-normalized mean absolute EE: 


\begin{align}
\mu_{i,\sigma}^* = \mu_i^* \frac{\sigma_{X_i}}{\sigma_Y}.
\end{align}

\noindent
This improvement is necessary for a consistent ranking of $X_i$. Otherwise, the ranking would be distorted by differences in the variation of the input parameters. The reason is that the input space constrains $\Delta$. If the input space is larger, the base value of $X_i$ can be changed by a larger $\Delta$.\\


\noindent
From the aforementioned set of drawbacks of the local derivate $D_i = \frac{\partial Y}{\partial X_i}$, there are two drawbacks remaining for the EE method. The first drawback is the missing direct link to the variation in $Var(Y)$. The second drawback is, that the choice of $\Delta$ is rather arbitrary if the derivative is not analytic. To this date, the literature has not developed convincing solutions to these issues.

In an attempt to establish a closer link between EE-based measures and Sobol' indices, \cite{kucherenko2009derivative} made two conclusions: The first conclusion is that there is an upper bound for the total index, $S_i^T$, such that
\begin{align}
S_i^T \leq \frac{\frac{1}{r} \sum_{j=1}^{r} {d_i^2}^{(j)}|}{\pi^2 \sigma_Y}.
\end{align}
This expression makes use of the squared EE. In light of this characteristic, the use of $\sigma_i$ as measure that aims to include the variation of $d_i^{j}$ appears less relevant. Yet, this rescaling makes the interpretation more difficult. The second conclusion is that the Elementary Effects method can lead to false selections for non-monotonic functions. This is also true if functions are non-linear. The reason is linked to the aforementioned second drawback, the arbitrary choice of step $\Delta$. More precisely, depending on the sampling scheme, $\Delta$ might be random instead of arbitrary and constant. In both cases, $\Delta$ can be too large to approximate a derivative. If, for example, the function is highly non-linear of varying degree with respect the input parameters $\bold{X}$, $\Delta > \epsilon$ can easily distort the results. Especially, if the characteristic length of function variation is much smaller than $\Delta$.



\subsection{Correlated input parameters}

So far, all measures described in this chapter assumed uncorrelated input parameters. Typically, this assumption is not met by realistic applications as joint estimations of parameters tend to produce correlated estimates. This gives reason to expand sensitivity measures to a more general setting.\\

\noindent
Today, there are several recent contributions that deal with the extension of the Sobol' sensitivity measures to the setup with correlated inputs. For instance, estimators for two complementary sets of measures have been developed by \cite{kucherenko2012estimation} and \cite{mara2015non}.
On the other hand, the only contribution to the computation of EE-based screening measures for the correlated setup has been made by \cite{ge2017extending}. Some authors, e.g. \cite{Saltelli.2004}\footnote{See page 46.} even negate the necessity for expanded Elementary Effects due to overcomplexity. Obviously, this can lead to false results.\\

\noindent
As the computation time of the \cite{Keane.1994} model is considerable\footnote{On my machine, it takes approximately six seconds to compute the QoI. Given that Sobol' indices can require a five digits number of evaluations per input parameter, the computation time is not small.}, the thesis reviews \cite{ge2017extending} and computes measures derived from this contribution to the occupational choice model. The aim is to fix parameters as preparation for a potential quantitative GSA. The measures correspond to the sigma-normalized mean absolute Elementary Effects. The measure in \citeauthor{kucherenko2009derivative} is discarded because their results are not valid in the presence of correlations.\footnote{The basic reason is that the unique first- and higher-order Sobol' indices do not add to one if input parameters are correlated.}


\newpage
\bibliography{../../bibliography/literature}

\end{document}