\section{Introduction}
\thispagestyle{plain} % surpress header on first page

How certain is a model outcome? This question is the subject of the mathematical field of uncertainty quantification (UQ). It is hard to find an area that is untouched by uncertainty. Therefore, all kinds of agents, but especially decision-makers and researchers, would benefit largely from a better understanding of uncertainties. For economic studies, the need for UQ  has been identified for quite some time (\cite{Hansen.1996}; \cite{Canova.1994}; \cite{Kydland.1992}). However, the economic research practice has not evolved accordingly (\cite{Harenberg.2019}).\\

\noindent
This thesis' objective is to quantify the parametric uncertainty of the occupational choice model in \cite{Keane.1994} (KW94). This model is of particular interest because it is an early representative of the class of Eckstein-Keane-Wolpin models that helped to shape the field of \textit{structural microeconometrics} by tightly interlinking the theoretical model with empirical work (\cite{Rust.2014}). It precedes the model in \cite{Keane.1997} which leads to the well-known outcome that skill differences at age 16 account for $90\%$ of the total variation in lifetime utility.

Assuming that the model parameters are estimated, the parametric uncertainty refers to the impact of the computed standard errors on the variation of the model outputs. I estimate the parameters from simulated data.
The uncertain model output, or quantity of interest (QoI), is the effect of a 500 US-Dollar (USD) subsidy on annual tuition costs for higher education on the average years of education. The UQ is divided into two parts. The first part deals with the general uncertainty of the QoI that is caused by all parameters together. The second part tries to identify the parameters that do not influence the QoI's variation.\\

\noindent
I find the following results for the general uncertainty: the mean QoI is an increase of 1.5 years in average education. The QoI's uncertainty is relatively low with 0.1 years.\\

\noindent
The second part does not provide clear results. This is due to two properties of the model.
The first property is that the model is rather complex and large, with 27 input parameters. Therefore, the computation time prohibits me from selecting the most conclusive measures.
The second property is that the model parameters are correlated.


For a model with these properties, I find that only the work by \cite{ge2014efficient} potentially includes measures that are suited to the model. These measures base on Elementary Effects (EEs). They are computed based on random samples which follow specific patterns (\cite{campolongo2011screening}). An EE is defined as the local derivative of a function with respect to one argument. The EE does not assume that the change in the argument is infinitesimally small (\cite{Morris.1991}). The analysis of the measures by \cite{ge2014efficient} reveals that they can lead to arbitrary parameter rankings. However, building on their work, I develop redesigned measures and demonstrate that they can precisely apportion the parametric uncertainty for a linear test function with correlated input parameters.

Nonetheless, I am not able to develop recommendations about which parameters to fix. The reasons are twofold, and they are not specific to correlated functions but specific to EE-based measures in general. First, it is unclear which EE-based measures to use. Conceptually, the EE can quantify the effect of the uncertainty of one parameter on the level of a QoI. However, the effect of EE-based measures on the variation of a QoI is not well-understood. Therefore, there is no consensus about which measure to use (\cite{campolongo2007effective}; \cite{kucherenko2009derivative}; \cite{Smith.2014}). Yet, I find some evidence that sigma-normalised EEs can lead to sensible results.
Second, the literature does not develop convincing cut-off criteria for declaring a parameter non-influential.

An additional result is that there is a high variance between the EEs computed from the data generated by two different sampling schemes. These schemes are the trajectory (\cite{Morris.1991}) and radial design (\cite{saltelli2010variance}). They differ in the way they generate the changes in the function arguments that are necessary for computing the local derivatives. The results indicate that these changes are too large to capture the variation of the KW94 model. This finding is in line with \cite{kucherenko2009derivative}.\\

\noindent
The structure of this thesis is as follows. I introduce the UQ framework and present the most important concepts in section 2. I review the applications of UQ in the economic literature in section 3. The following section reviews the EEs for correlated input parameters in \cite{ge2017extending} and develops the redesigned EEs. I also demonstrate my ability to replicate their results, and I validate my conceptual analysis of their measures as well as the redesigned measures with a linear test function. Section 5 introduces the occupational choice model by \cite{Keane.1994} and presents the estimation results for the input parameters.\footnote{The implementation of the computational model and the model estimation relies on the software packages \citetalias{Respy-Stenzel.2019} and \citetalias{Gabler.2019}.} Section 6 shows the results for the UQ.\footnote{I am responsible for the entire implementation of all applied UQ measures. The test coverage for these methods is $100\%$. All results can be found in the \citetalias{Stenzel.2020}.} Section 7 discusses the findings and identifies topics for future research. Section 8 concludes.





