\section{Introduction}
\thispagestyle{plain} % surpress header on first page

Structure: Need for UQ (incl. SA) - GSA - Importance measures - Dependent inputs - Goal of the thesis - Structure en passant

\textsc{\LARGE \bf [}\textit{Uncertainty quantification} (UQ) studies the components of a mathematical model that might contribute to its discrepancy from the real world. Modeling choices that cause such a discrepancy may be made because aspects of the real world are unknown to the modeler or because he intentionally chooses to simplify the model. Reasons for the latter can, for instance, be an emphasis on specific aspects of reality, algebraic ease, computation time, or numeric. UQ is a toolbox of methods that can help a researcher to reflect upon potential deficiencies of his model and to avoid the risk of overconfidence in his forecasts. Therefore, UQ is not only useful for improving the model but also crucial for providing differentiated, realistic forecasts, and well-founded thinking about useful policies.

In Economics, but of course also in other quantitative disciplines, one major challenge for using models to understand and quantify real-world phenomena, mechanisms, and effects is that model input parameters are often not well-known. In this case, usually, only some parameters that can describe a (joint) probability distribution of the model input parameters are available to the researcher. Consequently, the model outputs or \textit{quantities of interest} (QoIs) inherit this lack of knowledge from the input parameters. Put differently, the uncertainty in the input parameters is propagated through the model towards the QoIs. This gives rise to important questions like "Given the uncertainty of the input parameters, what is the probability distribution of the model outputs?". For instance, a model might predict an interesting outcome given the means of the parameter estimates. However, there might be a considerable probability that the model predicts an entirely different outcome. Of course, such findings should be reported if possible. Other important questions are "To what extent does the uncertainty of one or more specific input parameters contribute to the uncertainty of some QoI?" and "In which direction does each parameter affect the QoI globally?". These questions aim to inspect the influence of each parameter's or parameter group's uncertainty on the uncertainty of the respective QoI. They are also posed to investigate what effect higher-probability parameter values besides the usual measures of location can have on the QoI. Answers to these questions can then be used to reduce the model uncertainty in two ways: On the one hand, by devoting additional research to important input parameters that have a large influence on the QoI and its uncertainty and, on the other hand, they allow making informed decisions on whether or not to consider the uncertainty of less important input parameters or to even simplify their representation in the model. 
These questions are a subset of what is covered by the discipline of UQ. The field provides the suitable methods to answer the above questions profoundly. Their importance implies that these methods should be an essential part of every quantitative, model-based research and practice.

The goal of this thesis is to answer the above questions for an important and well-known economic model, thereby providing a showcase for the application of uncertainty quantification to Economics.

In section 2, I outline the general discipline of uncertainty quantification and present the subfields on Parameter Uncertainty that are featured in this thesis. These subfields are uncertainty propagation, \textit{global sensitivity analysis} (GSA) and surrogate models. Uncertainty Propagation is the construction of probability distribution functions for QoIs by propagating the model input uncertainty. GSA contains measures that indicate how much of the QoIs uncertainty can be attributed to specific parameters and parameter groups. The thesis employs 

' indices and univariate effects. uncertainty propagation and GSA can be computed in different ways. Besides Monte Carlo and quasi-Monte Carlo methods, one can also use surrogate models. A surrogate model substitutes the computation of a QoI via a model by approximating the QoI with one explicit, algebraic function of the model input parameters. They are used to save computation time if the evaluation of the original model is computationally expensive. This thesis uses both (quasi-)Monte Carlo methods and surrogate models to compute each result.

Section 3 gives an overview of the economic literature that uses UQ methods. It shows that, typically, economic research does not address the questions mentioned above, as the respective literature is small. Exceptions are \cite{Harenberg.2019} [Name other authors].

Section 4 presents the model for which the UQ is implemented. It is the well-known Dynamic Discrete Occupational Choice model developed by \cite{Keane.1994} in the field of labour economics. The QoI is the effect of a \$2000 subsidy on occupation choices. This model output is chosen because the result is of high relevance for policy-makers and thus well-suited to illustrate the benefits of performing a structured UQ in quantitative studies of economic models that may have important real-world implications. 

In Section 5, the results for the uncertainty propagation and global sensitivity analysis using Monte Carlo and quasi Monte Carlo methods are presented. [Add one sentence for the results.]

Section 6 shows the results for the same measures but computed by using a surrogate model. [Add one sentence for the results.]

Section 7 compares the two approaches and discusses the results. [Add two to three sentences for the results.]

Section 8 offers conclusions and indicates directions for future research.\textsc{\LARGE \bf ]}









