\section{Discussion}
\thispagestyle{plain}  % surpress header on first page

This chapter discusses the UQ's findings, its value and implications. The first part deals with the more general results from the uncertainty analysis. Then, I address the qualitative GSA and identify important challenges for future research. This part of the discussion is organised along the areas measures, sampling and estimation.\\

\noindent
I analyse the uncertainty of the effect of a 500 USD subsidy on annual tuition costs for higher education on the average years of education caused by the parametric uncertainty in the model of occupational choice in \cite{Keane.1994}. I find that the mean effect is an increase of 1.5 years in education for the whole population. This result is artificial because the parametric uncertainty is derived from simulated data. Assuming this is a real result would imply that this subsidy is a promising idea. As human capital is the backbone of modern economies, the subsidy could result in a significant economic upswing (\cite{mincer1984human}). Apart from that, education is linked to various other positive effects, for example on health (\cite{heckman2018returns}) and societal development (\cite{helliwell1999education}). Yet, regarding the monetary costs and that, as also noted by \citeauthor{heckman2018returns}, a prolonged education might not be the optimal choice for each individual a careful cost-benefit analysis would become necessary.

I also find that the input uncertainty accounts for a standard deviation of 0.1 in the QoI. This is a relatively small level of variation. Abstracting from the model, these two basic results are important for economists in general. They do not only help us to evaluate our model-based outcomes but they can also be helpful when we communicate with actors in the political arena. Indicating these results to, for example, important politicians and journalists proves a high level of sophistication and reflection. Therewith, our work may be received with higher regards and it would become more impactful. Of course, the basis is that the economic results are relatively certain.\\

\noindent
The qualitative GSA includes three findings.
First, it confirms the conceptual analysis that the two EEs in \cite{ge2017extending} have two crucial disadvantages. The first is that the independent EE is not unaffected by the correlations between parameters. In fact, it deflates the impact of inputs according to their level of correlations with other parameters. As both parameters, the independent and the full EE have to be interpreted jointly, one can not fix any parameter because this would require a reliable independent EE close to zero. The second drawback is that the draws in the numerator are in sample space and the draws in the denominator are in unit space. The analysis based on the redesigned EEs shows that it is not entirely clear which measure actually is adequate for factor fixing. However, the correlated and uncorrelated sigma-normalised mean absolute EEs, $\mu_\sigma^*$ are the only measures that are not detached from the scale of the QoI's standard deviation, $\sigma_Y$. This indicates that these measures are not only able to include the effect of the variation of $X_i$ on the level of $Y$ but also the effect on the variation of $Y$. I do not use this measure to recommend a fixing of any input parameter. The reason is that even for uncorrelated parameters, there is no consensus in the literature about what screening measures to choose, what cut-off criteria to use and what the particular links to the quantitative measures are. For example, \cite{campolongo2007effective} and \cite{ge2017extending} prefer the joint use of $\gamma^*$ and $\sigma$, \cite{kucherenko2009derivative} favour $(\frac{1}{r} \sum_{j=1}^{r} {d_i^2}^{(j)})/\pi^2 \sigma_Y$ and \cite{Smith.2014} highly recommends $\gamma^*_{\sigma}$.\\

\noindent
Second, the radial design generates higher EEs compared to the trajectory design for all parameters. In my view, the reason is that the radial design can generate steps $b-a$ as large as the whole sample space. This is a disadvantage for non-linear models with high variations. Furthermore, quantitative measures like the Sobol' indices do not consider such large differences. As these measures are variance-based, all draws are compared with the mean. This explanation is in line with the conceptual analysis in \cite{kucherenko2009derivative}. However, \cite{campolongo2007effective} find that the radial design is more precise in ranking the input parameters according to their effect on the variation in $Y$. Thus, it would be interesting to analyse these differences more closely. In any case, it is important to develop a sampling scheme that generates smaller steps $b-a$.\footnote{A possible path could be the following sketch for a hybrid design in unit space: the first row contains 0.5 for each column. Each step is a random draw in [0, 0.5]. Copy the changed draws to the next row as in the trajectory design and then shuffle each column to have equiprobable interactions for each parameter.}\\

\noindent
Third, the results indicate two potential interactions between maximum likelihood estimation and sensitivity analysis for correlated input parameters. The interactions require that the QoI is closely linked to the model observables $\pmb{\mathcal{D}}$.

Firstly, the correlated EE, $d_i^{c}$, is smaller than the uncorrelated EE, $d_i^{u}$, because parameters that have similar effects on observables $\pmb{\mathcal{D}}$ tend to be negatively correlated and those with opposite effects tend to be positively correlated. Therefore, the correlations "stabilise" the QoI against changes of specific input parameters.

Secondly, one observation from comparing the non-normalised measures to the sigma-normalised measures is that parameters that have a larger impact on the level of $Y$ tend to have a smaller impact on the variation in $Y$. This could be linked to a property of maximum likelihood estimation. The estimation implies larger standard errors for parameters with a smaller impact on the observables because they have a smaller influence on the probability of the observables. Consequently, a larger range of values for these less impactful parameters can potentially generate observables $\pmb{\mathcal{D}}$. Therefore, there might be an inverse relationship between ranking high with respect to measures that quantify the impact on the level of $Y$ and measures that deal with the variation in $Y$.

These arguments are rather speculative. It would be interesting if additional research could either confirm or contradict these hypotheses.
